\documentclass{article}

\usepackage[margin=1.5cm, includefoot, footskip=30pt]{geometry}
\setlength{\parindent}{0em}
\setlength{\parskip}{1em}
\renewcommand{\baselinestretch}{1}

\usepackage{amsmath}
\usepackage{amssymb}
\usepackage{hyperref}

\usepackage{minted}
\usemintedstyle{tango}

\title{Resultant theory. A brief overview.}

\begin{document}
\maketitle

Recently I am working on a project to explore the memory size effect of players
in the iterated prisoner's dilemma. While working on my project, I have came
across a number of multivariate systems. There are two things I need to know
about the systems:

\begin{itemize}
    \item do the polynomials of each system have common roots?
    \item and if they do, could we identify those roots.
\end{itemize}

To answer these two questions it was decided that I was going down the line of
the \textbf{resultant theory.} The resultant is a function that can help us 
to:

\begin{itemize}
    \item identify whether the system has a common root
    \item and can also help us to find those roots.
\end{itemize}

Firstly let explain in detail what the resultant is. Assume that \(p\) and \(q\)
are two polynomials the can be factored into linear factors,

\begin{eqnarray*}
    p(x) = & a_{0} (x - r_{1}) (x - r_{2}) \cdots (x - r_{m}) \\
    q(x) = & b_{0} (x - s_{1}) (x - s_{2}) \cdots (x - s_{n})
\end{eqnarray*}

then the \textbf{resultant} \(R\) of \(p\) and \(q\) is defined as,

\begin{equation*}
    R =a_{0}^{n}b_{0}^{m}\prod_{{i=1}}^{m}\prod_{{j=1}}^{n}(r_{i}-s_{j})
\end{equation*}

From the definition, it is clear that the resultant will equal zero if and only
if \(p\) and \(q\) have at least one common root. Thus, identifying whether
common roots exist for a system becomes trivial. Though calculating the resultant
as the products of the roots of each polynomial can be computational expensive.

An explicit formula for the resultant as a determinant was given by \textbf{Sylvester} in 1840
\url{http://www.tandfonline.com/doi/abs/10.1080/14786444008649995}. Assume that
\(p\) and \(q\) are given by,

\begin{eqnarray*}
    p(x) = & \displaystyle\sum_{i=0}^{m} a_i x^i \\
    q(x) = & \displaystyle\sum_{i=0}^{n} b_i x^i
\end{eqnarray*}

where \(deg(p) = m\) and \(deg(q) = n\). The Sylvester matrix is defined as
the \((m+n) \times (m+n)\) matrix,

\[
    \left|\begin{matrix}
        a_{0} & a_{1}& a_{2} & \ldots &a_{m}     & 0     & \ldots & 0 \cr
        0     & a_{0}& a_{1} & \ldots &a_{{m-1}} & a_{m} & \ldots & 0 \cr
              &      & \ddots&.&.&.&.&.\cr
        0&0&.&.&.&.&.&a_{m}\cr 
        b_{0}&b_{1}&b_{2}&\ldots&b_{n}&0&\ldots&0 \cr
        0&b_{0}&b_{1}&\ldots&b_{{n-1}}&b_{n}&\ldots&0 \cr 
        &&\ddots&.&.&.&.&.\cr
        0&0&.&.&.&.&.&b_{n}\cr
    \end{matrix}\right|
\]

in which there are \(n\) rows of \(p\) coefficients, \(m\) rows of \(q\) coefficients,
and all elements not shown are zero. It was proven that the determinant of the
Sylvester's matrix is equal to the determinant. The Sylvester's matrix is
implemented in the python library Sympy. Thus a numerical example is considered
using the Sympy library,

\begin{minted}
    [
    autogobble=true,
    framesep=2mm,
    fontsize=\normalsize,
    ]
    {python}
>>> import sympy as sym
>>> from sympy.polys import subresultants_qq_zz

>>> x = sym.symbols('x')

>>> p = x ** 2 - 5 * x + 6
>>> q = x ** 2 - 3 * x + 2

>>> matrix = subresultants_qq_zz.sylvester(p, q, x)
>>> matrix
Matrix([
[1, -5,  6, 0],
[0,  1, -5, 6],
[1, -3,  2, 0],
[0,  1, -3, 2]])
>>> matrix.det()
0
\end{minted}

An alternative matrix formulation of the resultant was given by Bezout during the
eighteenth century. The Bezout formulation was then reformulated by Cayley in 1865
and this is the second formulation discussed here, we will refer to it as the 
\textbf{Bezout Caley} formulation.

Consider again the polynomials \(p\) and \(q\) and let
\(d_{\text{max}} = max(degree(p, x), degree(q, x))\). The construct the following
polynomial,

\begin{equation*}
\Delta(x, a) = \left|\begin{matrix}
        p(x) & q(x)\cr
        p(a)& q(a)
        \end{matrix}%
        \right|
\end{equation*}

where \(\alpha\) is a new variable and \(p(a)\) stands for uniformly replacing
\(x\) by \(\alpha\) in \(p\). Making \(x = \alpha\) would make \(\Delta = O\)
which means that \(x – \alpha\) divides \(\Delta\). The polynomial is called the
Bezout Cayley polynomial, is of degree \(d_{\text{max}} - 1\) and is defined as,

\[\delta(x, a) = \frac{\Delta(x,a)}{x-a}\]

The polynomial is symmetric in \(x\) and \(\alpha\). Every common zero of \(p(x)\)
and \(q(x)\) is a zero of \(\delta(x, \alpha)\) no matter what value \(\alpha\) has;
thus at a common zero of \(p\) and \(q\), the coefficient of every power product
of \(\alpha\) in \(\delta(x, \alpha)\) must be 0.

Before moving forward let us consider a numerical example.

\begin{minted}
    [
    autogobble=true,
    framesep=2mm,
    fontsize=\normalsize,
    ]
    {python}
>>> import imp
>>> bezout = imp.load_source('bezout', '../src/cayley_bezout.py')
>>> p(x)
x**2 - 5*x + 6
>>> q(x)
x**2 - 3*x + 2

>>> a = sym.symbols('alpha')
>>> matrix = sym.Matrix([[p(x), q(x)], [p(a), q(a)]])
>>> matrix
Matrix([
[        x**2 - 5*x + 6,         x**2 - 3*x + 2],
[alpha**2 - 5*alpha + 6, alpha**2 - 3*alpha + 2]])
>>> bezout_polynomial = (matrix.det() / (x - a)).factor().collect(a)
>>> bezout_polynomial.expand()
2*alpha*x - 4*alpha - 4*x + 8
\end{minted}

Thus we have  a \(d_{max} - 1\) equations by equating the coefficients of 
the power products of \(\alpha\) to zero. Treating \(x ^0, x^1, \dots, x^{d_{max} -1}
\) as unknowns we retrieve \(d_{max} - 1\) equations in \(d_{max} - 1\) unknowns.

Theses will have a common root if and only if the the determinant of the
coefficient matrix is equal to 0, this \(d_{max} \times d_{max}\) matrix is
the Bezout Cayley matrix.

\begin{minted}
    [
    autogobble=true,
    framesep=2mm,
    fontsize=\normalsize,
    ]
    {python}
>>> bezout.cayley_bezout_matrix(p, q, x)
Matrix([
[-4,  2],
[ 8, -4]])
>>> bezout.cayley_bezout_matrix(p, q, x).det()
0
\end{minted}

Note that Bezout Cayley matrix for most cases, as shown in our example, is smaller
than the Sylverster's. So computational the Bezout Cayley can be favoured over
Sylverster's. Both formulation though can handle only up to a 2 variables system.

\textbf{Dixon} in 1908 showed how the Bezout Cayley ca be extended to $n + 1$ polynomials
in $n$ variables. Thus following a similar manner to that of Bezout Cayley,

\[
    \Delta(x_1,\dots, x_n, \alpha_1, \dots, \alpha_n) = \left|\begin{matrix}
        p_1(x_1,\dots, x_n) & \dots &p_n(x_1,\dots, x_n) \cr
        p_1(\alpha_1,\dots, x_n) & \dots &p_n(\alpha_1,\dots, x_n) \cr
        p_1(\alpha_1,\dots, \alpha_n) & \dots &p_n(\alpha_1,\dots, \alpha_n) 
        \end{matrix}\right|
\]

\[\delta = \frac{\Delta(x_1,\dots, x_n, \alpha_1, \dots, \alpha_n)}{((x_1 - \alpha_1) \dots (x_n - \alpha_n))}\]

The polynomials \(\delta\) is called the Dixon's polynomial and the Dixon's
matrix is constructed as,

A numerical example for Dixon:

\begin{minted}
    [
    autogobble=true,
    framesep=2mm,
    fontsize=\normalsize,
    ]
    {python}
>>> dixon = imp.load_source('dixon', '../src/dixon.py')
>>> y = sym.symblos('y')
>>> p = sym.lambdify((x, y), x + y)
>>> q = sym.lambdify((x, y), x ** 2 + y **3)
>>> h = sym.lambdify((x, y), x ** 2 + y)

>>> dixon = dixon.DixonResultant([p, q, h], (x, y))
>>> poly = dixon.get_dixon_polynomial()
>>> matrix = dixon.get_dixon_matrix(polynomial=poly)
>>> matrix.det()
0
\end{minted}

Now that I 
This work in more details developed in the following piece of work which is
no a pre print paper and under reviewing process. The pre print can be found
on arXiv:[https://arxiv.org/abs/1712.07640](https://arxiv.org/abs/1712.07640).
Furthermore, the code for all the numerical experiments held are
only and on github:
[https://github.com/drvinceknight/Evolutionary-game-theoretic-Model-of-Rhino-poaching](https://github.com/drvinceknight/Evolutionary-game-theoretic-Model-of-Rhino-poaching).
Note that all source code has been documented and tested
properly.

\end{document}