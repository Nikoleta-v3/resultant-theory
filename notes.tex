\documentclass{article}

\usepackage[margin=1.5cm, includefoot, footskip=30pt]{geometry}
\setlength{\parindent}{0em}
\setlength{\parskip}{1em}
\renewcommand{\baselinestretch}{1}

\usepackage{amsmath}
\usepackage{amssymb}
\usepackage{hyperref}

\usepackage{minted}
\usemintedstyle{tango}

\title{Resultant theory. A brief overview.}

\begin{document}
\maketitle

Recently while I working on my research a came across a problem.
I was faced with a number of multivariate systems and there are two things I need
to know for these systems,

\begin{itemize}
    \item do the polynomials of each system have common roots?
    \item and if yes could we identify them;
\end{itemize}

In order to address these questions and move on with my research it was decided
that I was going down the line of the resultant theory! The resultant initially help
us identify whether the system has a common root and furthermore we could use
resultant theory to find those roots as well. Here I will only talk about the
first usefulness of the resultants.

If \(p\) and \(q\) are two polynomials the can be factored into linear factors,

\begin{eqnarray*}
    p(x) = & a_{0}(x-r_{1})(x-r_{2})\cdots(x-r_{m}) \\
    q(x)= & b_{0}(x-s_{1})(x-s_{2})\cdots(x-s_{n})
\end{eqnarray*}

then the \textbf{resultant} \(R\) of \(p\) and \(q\) is defined as,

\begin{equation*}
    R =a_{0}^{n}b_{0}^{m}\prod_{{i=1}}^{m}\prod_{{j=1}}^{n}(r_{i}-s_{j})
\end{equation*}

From the definition, it is clear that the resultant will equal zero if and only
if \(p\) and \(q\) have at least one common root. Thus if we want to know
if common roots exist we only have to calculate the resultant and test whether
it is equal to zero.

Calculating the resultant in this way can be expensive. An explicit formula for
the resultant as a determinant was given by Sylvesterin 1840
\url{http://www.tandfonline.com/doi/abs/10.1080/14786444008649995}.

Let us consider that \(p\) and \(q\) are,

\begin{eqnarray*}
    p(x) = & \displaystyle\sum_{i=0}^{m} a_i x^i \\
    q(x) = & \displaystyle\sum_{i=0}^{n} b_i x^i
\end{eqnarray*}

where  \(deg(p) = m\)and \(deg(q) = n\). The Sylvester matrix of \(p\) and \(q\)
is the \((m+n) \times (m+n)\) matrix,

\[
    \left|\begin{matrix}
        a_{0} & a_{1}& a_{2} & \ldots &a_{m}     & 0     & \ldots & 0 \cr
        0     & a_{0}& a_{1} & \ldots &a_{{m-1}} & a_{m} & \ldots & 0 \cr
              &      & \ddots&.&.&.&.&.\cr
        0&0&.&.&.&.&.&a_{m}\cr 
        b_{0}&b_{1}&b_{2}&\ldots&b_{n}&0&\ldots&0 \cr
        0&b_{0}&b_{1}&\ldots&b_{{n-1}}&b_{n}&\ldots&0 \cr 
        &&\ddots&.&.&.&.&.\cr
        0&0&.&.&.&.&.&b_{n}\cr
    \end{matrix}\right|
\]

in which there are \(n\) rows of \(p\) coefficients, \(m\) rows of \(q\) coefficients,
and all elements not shown are zero. The resultant is the determinant
of the Sylvester matrix.

The Sylvester's resultant is implemented in the python library
Sympy. Thus here I will give you a numerical example. Note that I have also
implemented the Sylvester's resultant and my code can be found here:
but for the example I will be using the implemented version of Sympy.

\begin{minted}
    [
    autogobble=true,
    framesep=2mm,
    fontsize=\normalsize,
    ]
    {python}
>>> import sympy as sym
>>> from sympy.polys import subresultants_qq_zz

>>> x = sym.symbols('x')

>>> f = x ** 2 - 5 * x + 6
>>> g = x ** 2 - 3 * x + 2

>>> matrix = subresultants_qq_zz.sylvester(f, g, x)
>>> matrix
Matrix([
[1, -5,  6, 0],
[0,  1, -5, 6],
[1, -3,  2, 0],
[0,  1, -3, 2]])
>>> matrix.det()
0
\end{minted}

Note that the Sylvester's resultant is very useful but can only handle up to 
2 variables. An alternative matrix formulation was given by Bezout during the
eighteenth century. The Bezout formulation was then reformulated by Caley in 1865
and this is the second formulation we will discuss here. I will calling it
the Bezout Caley formulation.

Let us consider again the univariate polynomials \(p\) and \(q\). Let the 
\(d_{\text{max}} = max(degree(p, x), degree(q, x))\). Consider the polynomial,

\begin{equation*}
\Delta(x, a) = \left|\begin{matrix}
        p(x) & q(x)\cr
        p(a)& q(a)
        \end{matrix}%
        \right|
\end{equation*}

where \(\alpha\) is a new variable and \(p(a)\) stands for uniformly replacing
\(x\) by \(\alpha\) in \(p\). Making \(x = \alpha\) would make \(\Delta = O\)
which means that \(x – \alpha\) divides \(\Delta\). Thus the Bezout Cayley 
polynomial of degree \(d_{\text{max}} - 1\)is defined as,

\[\delta(x, a) = \frac{\Delta(x,a)}{x-a}\]

The polynomial is symmetric in \(x\) and \(\alpha\). Every common zero of \(p(x)\)
and \(q(x)\) is a zero of \(\delta(x, \alpha\) no matter what value \(\alpha\) has;
thus at a common zero of \(p\) and \(q\), the coefficient of every power product
of \(\alpha\) in \(\delta(x, \alpha\) must be O.

Thus we have  a \(d_{max} - 1\) equations by equating the coefficients of 
the power products of \(\alpha\) to zero. Treating \(x ^0, x^1, \dots, x^{d_{max} -1}
\) as unknowns we retrieve \(d_{max} - 1\) equations in \(d_{max} - 1\) unknowns.
They will have a common root if and only if the the determinant of the
coefficient matrix is equal to 0, this \(d_{max} \times d_{max}\) matrix is
the Bezout Cayley matrix.

A numerical example:

\begin{minted}
    [
    autogobble=true,
    framesep=2mm,
    fontsize=\normalsize,
    ]
    {python}
>>> f = sym.lambdify(x, x ** 2 - 5 * x + 6)
>>> g = sym.lambdify(x, x ** 2 - 3 * x + 2)

>>> matrix = subresultants_qq_zz.bezout(f(x), g(x), x)
>>> matrix
Matrix([
[ 8, -4],
[-4,  2]])
>>> matrix.det()
0
\end{minted}

\end{document}